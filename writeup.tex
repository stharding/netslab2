\documentclass[]{article}

\usepackage[margin=.885in, paperwidth=8.5in, paperheight=11in]{geometry}
\usepackage{titlesec}
\usepackage[normalem]{ulem}
\usepackage{amssymb} 
\usepackage{amsmath}
\usepackage{booktabs}
\usepackage{algorithm}
\usepackage{algorithmic}
\usepackage{color}
\usepackage{graphicx}
\usepackage{listings}
\usepackage{hyperref}
\usepackage{adjustbox}
\usepackage{tikz}
\usepackage{adjustbox}
\usepackage{fp}
\usepackage{listings}

\titleformat{\section}{\normalfont\Large\bfseries}{\thesection.}{.4em}{}
 
\setlength\parindent{0pt}
\setlength{\parskip}{5pt}

\begin{document}

\title{The Effectiveness of Selective Acknowledgments in a Lossy Network}
\author{Stephen Harding}

\maketitle

\begin{abstract}\bf\normalsize
In a network where packet loss is uncommon, the naive strategy of only
acknowledging packets when all previous packets in the sequence have been
received and acknowledged works fairly well. However, in a network 
where packet loss is ubiquitous, this strategy can lead to excessive and
gratuitous retransmission of packets. e.g., the recipient is missing a packet
and has received several packets that are after the missing packet but never
sends an acknowledgement to the sender for these packets. The sender ends up
retransmitting all of the unacknowledged packets because it has no way of knowing
that the recipient was actually only missing a single packet. The standard way
of addressing this problem is selective acknowledgements \cite{floyd1996tcp} 
\cite{floyd2000extension}. This  allows the recipient to acknowledge ranges of 
packets which has the effect of notifying the sender of the missing packets so 
they can retransmit only the necessary packets.

In this paper, we examine the effectiveness of selective acknowledgements in
a network environment where there is a high rate of packet loss. 
\end{abstract}\normalfont

\section{Introduction}
To study the effectiveness of selective acknowledgements (\texttt{sack}) we perform a 
full factorial experiment in which we hold constant a rate of packet loss and
vary the network latency. We measure retransmission rates with selective
acknowledgments on and off for each latency setting.

The rest of this document is laid out as follow. We describe the experimental 
setup in detail in section \ref{setup}. In section \ref{results} we present
the results of the experiments. In section \ref{discussion} we discuss the
results presented in section \ref{results} before concluding in section
\ref{conclusion}.

\section{Experimental Setup}\label{setup}
We perform a full-factorial experiment with two factors: network latency and
selective acknowledgement. The selective acknowledgement factor has two levels,
either on or off. For the network latency factor, we consider several different
latencies. The experiment is conducted on a network of three virtual machines 
where there are two endpoints on different subnets and a router that connects 
the two. The network delay and latency is simulated with \texttt{tc} rules on 
the router. A client program is run on one endpoint (a FreeBSD machine) and a 
server is run on the other (an Ubuntu machine). The client sends a random string 
of 1,000,000 characters to the server and the server sends back an MD5 checksum 
of the transmitted string. Since only the server is receiving large amounts of 
data, we only vary the state of the selective acknowledgement \texttt{TCP} option 
on the server. 

The latency factor has the following levels: 0ms, 5ms, 5ms $\pm$ 2ms, 50ms, 50ms 
$\pm$ 10ms, 100ms $\pm$ 40ms, and 500ms. Levels with a $\pm$ are varied by the 
amount after the $\pm$ and are normally distributed. Levels without a $\pm$ have 
no variation in the latency other than the variation inherent in the actual network.

\texttt{TCP} dumps are recorded on the client machine in order to
measure the rate of retransmitted packets. Each treatment is measured 30 times. 
The packet loss is implemented by a \texttt{tc} rule on the router: 
\texttt{tc qdisc change dev eth1 root netem loss 25\% 25\%}


\subsection{Bias and Distortion}
We take two measures to minimize bias and distortion.
First, treatments are not blocked. The treatments are performed at random by
building a list of experiments to be performed and randomizing that list prior
to beginning the experimentation. Second, no other extraneous processes or 
network connections are allowed to run during any of the experiments. All
network communication to change the testbed is done in-between treatments.

\section{Results}\label{results}

Since there are 196 entries in the full factorial experiment we must restrict ourselves
to a few representative results. \footnote{The reader is invited to view the results 
in more detail by examining the raw data found at github.com/stharding/netslab.git} 
To analyze the results we inspect the \texttt{TCP} dumps and generate text files which
contain information about packet retransmission which we can then analyze with the
\texttt{R} statistical tools.

To begin, consider the 0ms delay. We find that difference in the mean rate of retransmission
when \texttt{sack} is varied is statistically significant with  a t-test \footnote{We
justify the use of a t-test here by making use of the standard normality tests available
in \texttt{R} (e.g. shapiro.test, qqnorm, qqline, etc.) The reader is welcome to reproduce
these results by obtaining the repository listed in footnote 1}, with the sample means of
107.8 and 125.4 for \texttt{sack} on and off respectively. We see a similar reduction in
retransmission in the 5ms case: 130.2, and 145.0. The benefit is slightly reduced in the 5ms 
$\pm$ 2ms case: 128.1, and 139.4. Interestingly, the 100ms $\pm$ 40ms case resulted in no
statistical significance but the 500ms with \texttt{sack} on performed better than 100ms 
$\pm$ 40ms with 139.3 and 150.4 respectively. Even more interesting is the following. The
treatment with 500ms with no variation and with \texttt{sack} on resulted in no statistical 
significance for a difference of means with each of the following treatments: 100ms $\pm$ 
40ms \texttt{sack} on, 50ms \texttt{sack} on and off, 50ms $\pm$ 10ms \texttt{sack} on and 
off, 5ms $\pm$ 2s \texttt{sack} off, and 5ms \texttt{sack} off. The most dramatic reduction 
in retransmission is the 500ms case: 139.3, and 212.6 with \texttt{sack} on and off 
respectively.

\section{Discussion}\label{discussion}
We can see from the results that selective acknowledgements can have a dramatic impact
on the number of retransmissions that are sent in a lossy network. Furthermore, it is clear
that selective acknowledgements enjoy the greatest benefit in a network environment where
the network latency is constant. In fact, as we saw in the previous section, with selective
acknowledgements enabled, even a very high delay is much better than a lower delay with
high variation.

\section{Conclusion}\label{conclusion}
Clearly selective acknowledgements are a beneficial \texttt{TCP} option in a lossy network.
While we did find some cases where enabling selective acknowledgements did not make an
obvious beneficial difference, we found no cases where enabling selective acknowledgements 
performed worse than disabling them.

Future work would benefit from adding a factor to the experiment in which the levels would
vary congestion control parameters.

\bibliography{refs}{}
\bibliographystyle{plain}
\end{document}

